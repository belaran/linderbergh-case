
\subsection{Au chat et à la souris, à travers Manhattan}

\paragraph{} Equippé d'une seule voiture de fonction, un banal Fort T, avec dans le coffre un attirail de déguisement, les 
agents arrivent devant la demeure de \gls{jafsie}, après une route épuisante de près de 6h ou 7h\footnote{Test de Conduite pour 
le pilote, sinon -1 point de vie, causé par la fatigue à tous les passages}. Ils arrivent juste à temps, le vieux monsieur, 
clairement nerveux, mais décidé, vérouille sa porte et se prépare à partir.



\subsection{Le Cimetière Saint Raymond}

\paragraph{} Après une délicate filature de \gls{jafsie}, qui a été baladé dans presque tout Manhattan, avant d'être redirigé 
vers le Bronx, les investigateurs arrivent, à quelques centaines de mètres derrière le professeur à la retraite, en vue du 
Cimetière Saint Raymond. Sachant que le premier rendez vous avec les prétendus kidnappeurs a eu lieu dans le cimetière de Woodland,
les agents sentent intuitivement qu'ils parviennent enfin au lieu du rendez vous.

\paragraph{} Pénétrer dans le cimetière à la suite du vieux homme est moins évident que la filature, pourtant déjà peu aisé, qui
les a conduit ici. En effet, la zone est déserte, et \gls{jafsie} a fait déjà un beaucoup infernale, en faisant grincé la porte du 
cimetière. Si quelqu'uns surveille les lieux, suivre, l'air de rien, le vieux professeur ne passera pas inaperçu. 

\paragraph{} Heureusement, même si les investigateurs l'ignorent, la secte n'a rien mis en oeuvre pour surveiller l'arrivée de 
leur intermédiaire. Le vieux professeur ayant respecté les règles du jeux lors du premier rendez vous, ils sont donc désormais un 
peu moins méfiant.

\paragraph{} Quelques soient le stratagème de vos joueurs, leurs personnages devraient arrivées, peu après \gls{jafsie}, au coeur 
du cimetière, et ils ont trouveront refuge derrière une pierre tombale, ironiquement celle d'un Mafiosi mort il y a peu - Joe Cataria, 
pour observer l'échange. Quelques instants après leur arrivée, une église du Bronx sonne au loin minuit - comme quoi ce n'est pas toujours
un cliché, quand émerge de la partie Ouest du cimetière, très boisé, une silouhette. \gls{john}, le contact de \gls{jafsie} vient de 
faire son apparition.

\paragraph{} Tremblant, le vieil homme explique qu'ils n'ont pu réunir que 50 000\$, mais sous interlocuteur ne bronche pas et accepte 
l'argent et lui remet une note lui indiquant que le bébé est sain et sauf à bord d'un bateau, le \emph{Nelly}, à quai dans 
\href{http://en.wikipedia.org/wiki/Martha\%27s\_Vineyard}{Martha's Vineyard}. \gls{jafsie} court prévenir \gls{clg}, qui, lorsqu'ils ne 
trouveront pas le \emph{Nelly}, ira jusqu'à survoler la baie pour tenter de débusquer les kidnappeurs.

\paragraph{} Mais pour les investigateurs, le temps est venu d'honorer la mission secrète que leur a confié \gls{ejh}, suivre \gls{john} et
résoudre l'affaire. Une fois la transaction effectué, \gls{john} regarde le veil homme repartir au pas de course quelques instants, avant 
de lui même se replier vers la même partie boisée du cimetière d'où il a émergé.

\begin{figure}[h]
 \begin{center}
  \includegraphics[scale=0.5]{../handouts/a-la-poursuite-des-profonds-nb.jpg}
  \end{center}
\caption{A la poursuite de ``John''}
\end{figure}

\paragraph{} TODO: Distance / temps sur google maps

\subsection{Retour à la baie de Huson}

\paragraph{} Selon leur discrétion, les agents effectueront une filature, assez complexe, nécessitant quatre jets réussi de Discrétion :
\begin{itemize}
 \fltitem{un pour sortir du cimetière, sans se faire repérer par leur cible ;}
 \fltitem{ un autre pour traverser la Hutchison River Parkway - loin d'être une 4 voies d'aujourd'hui à l'époque et peu fréquenté à cette heure,
 toujours sans se faire voir ;} 
 \fltitem{ un troisième jet pour ne pas se faire repérer en traversant les maisons  entre Rohr Pl. et Senger Pl ;}
 \fltitem{ et enfin un ultime jet pour naviguer dans les entrepôt à poissons, dégageant une odeur plus que nauséabonde.}
\end{itemize}



\paragraph{} A chaque fois qu'un agent rate son jet de discrétion, \gls{john} peut tenter un test de Vigilance (il a 40\%). Si il réussit, il 
repère les agents et panique : il sort son arme à feu (TODO, 30\%), et effectue un \emph{tir de barrage} pour couvrir sa fuite, maintenant au 
pas de courses, vers le barque qui l'attend près des entrepôt.

\subsection{Abattre John}


\subsection{Perdre sa piste}

\paragraph{} Dans le cimetière, ils verront au moins vers où il aller , donc, reflexes de flics assermentés, ils pourront réveiller et interroger les 
habitants des maisons situés entre Rohr Pl. et Senger Pl. Après quelques maisons, un vieil homme, aussi alcoolic qu'insomniaque, aura vu la
silouhette de la \gls{john} se dirigeant vers les entrepots.

\paragraph{} De là, les agents pour répérer (Trouver Objet Caché) la piste boueuse du suspect et la suivre (Suivre une Piste) jusqu'au bord de 
l'eau.

\subsection{Suivre sa piste}

\paragraph{} Si les agents le suivent sans se faire remarquer, ils le verront monter à bord d'une petit barque, qu'il avait discrètement caché dans les 
arbres. \gls{john} jete le lourd sac de toile noire dans lequel il a placé les 50000\$ remis par\gls{jafsie}, et, le plus silencieusement possible, 
s'engage, à l'aide de rame s'engage sur le \href{http://en.wikipedia.org/wiki/Westchester\_Creek}{Westchester Creek}.

\paragraph{} Si ils suspectent leur présence (plusieurs test de discrétion raté du coté des agents, sans succès de sa part pour les repérer), il 
s'éloignera de la cote, mais se rapprochera de nouveau dès qu'il approchera trop de Clason's Point\footnote{Voir les 
\href{http://www.tentacules.net/index.php?id=1180}{Secrets de New York} pour en savoir plus sur Clason's Point et pourquoi un membre de \gls{osd}
se méfie naturellement de cet endroit}. 

\paragraph{} Sa prise distance lui aura été fortement inutile, les agents ne le perdront pas de vue, malgré l'obscurité\footnote{La nuit du 2 mai 1932 n'était 
qu'à \href{www.moonconnection.com/moon_phases_calendar.phtml}{3 nuits d'écart} de la nouvelle lune.}, et cette manoeuvre lui donnera l'impression erronée
qu'il a semet ses poursuivants.


\begin{wrapfigure}{r}{60mm}
\begin{center}
 \includegraphics[scale=0.2]{../handouts/westchester-creek-river.png}
\end{center}
\caption{Westchester Creek River}
\end{wrapfigure}


% bridge : www.panoramio.com/photo/29343659

\paragraph{} Suivre l'esquif alors que \gls{john} canote n'est pas très difficile, mais, alors plus il descend la rivière, plus les alentours deviennent 
relativement rural pour la région de New York. Les agents longent une rivière, caché par quelques arbres, au loin, on peut distinguer la route XXX
et le pont \href{http://en.wikipedia.org/wiki/Bronx–Whitestone\_Bridge}{Whitestone Bridge}, mais aucune voiture n'y circule à cette heure ci et 
force est d'admettre que le grand ouvrage de métal est plutôt lugubre vu d'ici.

\paragraph{} A l'exception du remoud de l'eau et de quelques hulement nocturne, le lieu est désert et semble presque sauvage. La fatigue et la 
tension de la filature continue de jouer des tours aux investigateurs qui, si ils ratent un test de SAN - gardez la caractéristique testé secrète, commence à
distinguer dans les ombres comme des silouettes, nombreuses et que vaguement humaines, qui semblent les observer.

\subsection{Rendez vous sous le Whitestone Bridge}


\begin{wrapfigure}{r}{65mm}
\begin{center}
 \includegraphics[width=64mm]{../handouts/bridge.jpg}
\end{center}
\caption{Whitestone Bridge}
\end{wrapfigure}

\paragraph{} Une fois le pont atteint, \gls{john} s'amarre sous le premier pylone, à 20 mètres de la rive et attend. Rapidement, un petit bateau 
de pêche apparait, le \emph{Nelly} que les investigateurs auront peut être remarqué lors de leur filature de \gls{jafsie} à travers Manhattan. Le
navire de pêche dispose de plusieurs sources de lumière, sous la forme de lampe-tempête accroché ici et là, ce qui permet aux agents d'observer la
scène de loin, avec les jumelles qu'ils auront bien évidemment penser à prendre.

\paragraph{} L'échange sera rapide, \gls{john} remet le sac noir contenant l'argent à un vieux marin barbu, et il se remet en route, toujours plus 
vers le sud. Il rentre chez lui, une maison délabré sur la rive de l'Hudson, dans le quartier de Throg's Neck, les investigateurs peuvent aisément 
continuer à lui suivre si ils le désirent.

\paragraph{} Alors que \gls{john} s'éloigne, le capitaine du Nelly disparait dans son bateau avec le sac noir. Il ressort rapidement avec un lourd
filet de pêche sur le dos. A l'aide d'un jet de Trouver Object Caché, les investigateurs observent la scène que le contenu du filet de pêche semble
se débattre !

\paragraph{} Le vieux marin jete son bardas à l'eau et dans la foulé relance son moteur pour repartir aussi en direction de Throg Neck. Si les 
investigateurs veulent sauver la pauvre femme - une des victimes de \gls{ams} gentillement donner à \gls{osd}, il ne disposent que de peu de temps. Tout investigateurs 
se lançant dans une tentative de sauvetage doit réussir près de 3 jet d'Athlétisme, en tant que Natation, pour arriver assez rapidement, pour sortir
la jeune femme de l'eau.

\paragraph{} Si l'agent ne réussit pas tout ses jets de dés, il ne peut que ramener le corps noyé de la pauvre femme - et encore seulement si il réussit
un test de Puissance pour soulever la frêle créature et le lourd filet de pêche. Avant de tenter la ramener à bon port, il ou ils devront aussi
découper le filet de pêche, assez lourd, qui leur rendrait la tâche impossible. 

\paragraph{} Un autre élément, bien plus effrayant, s'opposera au sauvetage de la jeune femme : un Profond. Le vieux marin n'a pas jeté la pauvre
demoiselle juste pour s'en débarasser, mais pour l'offrir aux Profonds. Une de ses créatures attendait donc sous le pont pour s'emparer d'elle - et
lui jeter un sort permet à la jeune femme de survivre à la noyade. 

\paragraph{} Si les investigateurs interviennent, le Profond, qui comme la plupart de ses comparses évitent la confrontation directe avec les humains,
essayera juste d'attraper sa \emph{promise} et la tirer vers les profondeurs. Une fois que les investigateurs auront ramené la demoiselle à la surface
et essayeront de la libérer de son filet de pêche - action auquelle elle n'opposera ni résistance ni assistance, elle est déjà inconsciente, le Profond
se glisse entre eux et tire subitement sur le filet de pêche pour la faire replonger.

\begin{wrapfigure}{l}{60mm}
\ovalbox{
  \scalebox{0.9}{
  \begin{minipage}{2in}
  \textbf{Et si les investigateurs n'assistent pas à l'échange ?}
  \begin{center}------\end{center}
  \paragraph{} Dans ce cas, le cadavre de la jeune femme, qui n'aurait pas survécu à la noyade
  malgré le sort du Profond, réapparaitra le lendemain, non loin de là. Comme les agents auront très probablement rapporté leur escapades dans le
  Bronx à \gls{ejh}, ce dernier ne manquera pas de faire le lien - ses assistants ont reçu ordre, juste après le rapport des investigateurs de lui 
  signaler tout rapport ou évènement étranges à New York et dans les environs. Néanmoins, l'information ne parviendra pas exactement au \gls{fbiHq},
  et les investigateurs n'auront vent de ceci que la nuit d'après, soit le 3 mai.
  \end{minipage}
  }
}
\end{wrapfigure}

\paragraph{} Avec un test de Vigilance réussit, les investigateurs pourront opposer un test de Puissance face au Profond et pourront peut être l'empêcher
de s'emparer de sa victime. En outre, ils pourront percevoir que \textit{quelque chose} se déplace sous eux (Au cas où vos agents se sentent
grisés de porter un badge du \textit{Bureau of Investigation} et décide de faire un \textit{carton}, rappelez leur l'effet regretable de l'eau
sur la poudre...).

\paragraph{} Le Profond n'insistera pas longtemps et cherchera à tout prix à éviter la confrontation, si ceux deux premières tentatives échouent ou
si les investigateurs arrivent rapidement à libérer la jeune femme du filet de pêche, il abandonne et s'enfonce dans le fond des eaux, pour rejoindre
l'Hudson puis l'océan.
