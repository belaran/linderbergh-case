\subsection{Introduction}
\subsection{Le Lindbergh Case}
\paragraph{} Ce scénario se déroule donc dans le cadre historique du \emph{\gls{tlc}}, comme l'a baptisé le
\gls{fbi}. La déroulement des actions, et l'implications des différents personnages historiques et la chronologie, y est 
autant que possible respecter, ce qui vous permettra de vous référer sans difficultés aux nombreuses sources disponibles
sur le sujet. Une bibliographie et une ``webliographie'' est présente en annexe à cette intention.
\paragraph{} Bien évidement, pour vous faciliter la lecture et la prise en main du scénario, un résumé de l'affaire, du
déroulement des évènements et des personnages clés est présent ci dessous. Si il devrait suffir à mener ce scénario sans
difficultés majeures, se plonger un peu plus en profondeur sur le contexte historique du scénario vous permettra d'apporter
vraisemblablement une touche de profondeur et d'authenticité supplémentaire appréciable à votre narration. Sans compter
évidement qu'une connaissance claire de cette dernière votre permettra plus aisément de répondre aux interrogations de 
vos investigateurs.
\begin{wrapfigure}{r}{60mm}
\ovalbox{
  \scalebox{0.9}{
  \begin{minipage}{2in}
  \textbf{Jouer avec ses investigateurs habituels}
  \tiny{
  \begin{center}------\end{center}
\paragraph{} Si vous souhaitez jouer ce scénario dans le cadre de votre campagne avec les personnages habituels de vos joueurs, il est peu probable
que vous puissiez facilement les transformer subitement en enquêteurs du FBI. Ceci ne forme pas néanmoins pas un réel problème, ni une
réelle difficulté.
\paragraph{} Si \gls{ejh} confie naturellement cette mission à des agents du \gls{fbi}, il peut tout autant la confier à des ``consultants 
externes'' du bureau - vos investigateurs peuvent avoir déjà été engagés par le bureau, probablement pour aider des agents empétré dans une affaire
quelques peu étrange. Que vos investigateurs y ait joués un rôle de démystificateur\footnote{Dévoilant, encore une fois, comme dans tout bon
épisode de \emph{Scoobidoo}, que le fantôme, c'était le directeur du musée.} ou au contraire aient usées de leurs connaissances des abominations
du Mythe de Cthulhu, ils n'en restent pas moins qu'ils ont été chaudement recommandé par des agents du FBI au directeur. 
\paragraph{} Peut être, ont ils aidés \gls{ctn}, le fidèle bras de droit de \gls{ejh} avant qu'il n'accède à un si haut échellons du Bureau. A 
moins qu'il n'est aidé, le numéro 3, Cartha (TODO) dans une affaire embarrasante. Quoiqu'il en soit, les investigateurs sont fortement recommandés
pour leur discernement et leur discrétion au directeur qui opte donc pour les utiliser, plutôt que ses propres agents, diminuant d'autant plus le
risque d'implication du Bureau.
  }
  \end{minipage}
  }
}
\end{wrapfigure}


\subsection{Les Secrets de New York}
\paragraph{} L'action du \gls{tlc} se déroulant essentiellement à New York City, et plus spécifiquement, dans sa banlieue célèbre 
du Bronx, le guide pour l'Appel de Cthulhu \gls{sny} a aussi été une inspiration importante pour ce scénario. Les deux principales
opposants des investigateurs en sont directement extraits:
\begin{itemize}
 \item \gls{ams}, un sorcier plusieurs fois centenaires, mais aussi un riche - mais discret, notable de New York est décrit amplement 
dans le guide \gls{sny}. Il est le ``véritable'' commanditaire du \emph{kidnapping} du jeune fils de \gls{clg} et le démasquer forme 
donc, du point de vue des investigateurs un objectif essentiel.
 \item \gls{osd}, très succintement évoqué dans le guide, joue ici un rôle de ``sous traitant'' pour \gls{ams} s'étant chargé pour 
son compte du \emph{kidnapping}.
\end{itemize}

\begin{wrapfigure}{l}{40mm}
\ovalbox{
  \scalebox{0.9}{
  \begin{minipage}{2in}
  \textbf{Et si les investigateurs ont déjà joué le scénario la Demi Lune ?}
  \begin{center}------\end{center}
  \paragraph{} Ce scénario, inclut dans \gls{sny}, base sa trame sur la découverte et la destruction du sorcier et de son culte 
par les investigateurs. Dans le cas, où vous sohaiteriez jouer ce scénario avec les mêmes investigateurs, il faudra opérer quelques
modifications au scénario:
\begin{itemize}
 \item les pistes du scénario menant à \gls{ams} mèneront à un membre de sa secte, dont les investigateurs pourront se rappeler suite
à leur précédente enquête sur le sinistre sorcier. Si ils avaient réussi à écrouer l'individu, ils se rappeleront que ce membre de la
secte avait réussi à échapper aux poursuites judiciaire, faute de preuves. Notez qu'un membre de la cour suprême ferait un bon remplaçant 
à \gls{ams}, car un tel personnage entrainerait autant de prudence que \gls{ams} chez \gls{ejh}. 
 \item les installations de la Morgens Institude (TODO page du guide) auront été vraisemblablement abandonnées à la suite des investigations
de vos joueurs, il faudra donc relocaliser ces derniers. Le guide \gls{sny} vous aidera dans cette tâche, mais, en guise de suggestion, 
\gls{ams}
  \item Si \gls{ams} n'avait pas survécu à sa rencontre avec vos personnages, il sera de retour pour ce scénarion, réincarné par l'un des survivants
de la secte. Evidament, la réssurection de ce dernier nécessitait de s'emparer du corps, sans éveiller les éventuels autorités, une tâche, où là 
encore, un juge de la Cour Suprême excellerait.
\end{itemize}
  \end{minipage}
  }
}
\end{wrapfigure}

\subsection{Les rêves eugéniques de Ambrose Morgens}

\paragraph{} Comme décrit dans \gls{sny} , \gls{ams} est sorcier aux services de \gls{nya} qui est obséder, depuis maintenant plusieurs
millénaires, par l'amélioration, par transformation relativement radicale, de l'espèce humaine. Les idées eugénistes sont à la mode dans 
les années vingts et l'on discute à tout bout de champs du ``problème'' du mélange des races. C'est ce genre de réflexion qui aboutira en
1924 (TODO, lien date) à la publication du National Immigration Act. Néanmoins, cette mode a beaucoup aidé le sorcier dans ses recherches,
lui apportant entres autres, et jusqu'à la crise de 1929, de nombreux financement.
\paragraph{} Il se sent, avec son comparse \gls{evk}, qu'il considère plus comme un outil qu'un collaborateur, près à réaliser l'ultime transformation, qui aboutira à la création d'une nouvelle espèce, supérieur, de l'homme. Pour réaliser ce dernier pas vers leur succès, ils ont décidés qu'il leur faudrait, en guise de cobaye, un bébé de quelques mois à peine, mais au potentiel génétique de forte qualité. 
\paragraph{} L'idée de \emph{kidnapper} le fils de \gls{clg} leur vint presque naturellement. En effet, \gls{clg} est une sorte de ``super start''
pour l'époque, même si son dégout pour les bains de foules à fini par retourner une partie de l'opinion, frustré de son manque de présence, contre
lui. Néanmoins, pour \gls{ams}, son fils, possédant clairement des gênes de grandeur, aux regards des exploits de son père, est un parfait cobaye
pour réaliser leur prochaine expériences.
\paragraph{} Il reste pour autant encore de réaliser le rapt en lui même, ce qui n'est pas simple, car la célébrité de l'aviateur lui assure la
collaboration de nombreuses forces de polices et agence - comme ça sera le cas, or la secte n'apprécie pas l'idée de risquer d'être découverte
lors de l'enquête qui suivra l'exécution du crime. C'est alors que le sorcier pense à \gls{osd}

\subsection{L'Ordre Esotérique de Dagon de Throg's Neck}

\subsubsection{L'alliance avec Ambrose Morgens}

\paragraph{} Au début des années 1930, \gls{ams} est rentrée en contact avec quelque uns des membres
de l'\gls{osd}, et a commencé avec eux un avantageux commerce. En échange des services de la secte, 
le sorcier leur fournit, sur une base régulière, des \emph{femelles} - comprendre des femmes qu'il a 
kidnappées pour ses propres besoins. Une fois ces dernières rescapés du laboratoire de son comparse, et 
créature, \gls{evk}.

\paragraph{} Néanmoins quand il ne s'agit plus de \emph{kidnapper} quelques jeunes adolescents brillant 
de la haute société New Yorkaise, mais le fils du plus grand héros des états unis (\gls{clg} est une véritable 
icône pour ses comtemporains), \gls{ams} décide d'utiliser le culte comme intermédiaire, sachant que sa 
relation avec eux est très tenue. Le culte ignore qui il est vraiment, car si il préfèrent rencontrer \gls{osd}
en personne, généralement dans les cimettières du Bronx, il prend toujours soin de masquer son visage, comme
sa voie, à l'aide de Magie.

\subsubsection{Profonds et reproduction}

\paragraph{} La communauté de Throg's Neck est essentiellement composée de vieilles familles de pêcheurs, 
qui se sont recyclés en garde maison quand les riches ont transformé, assez brièvement, le village en une
sorte de port de plaisance pour riche désoeuvrés dans les années 20.

\paragraph{} La communauté est entièrement noyauté par \gls{osd}, à l'insu des rares notables de New York qui
fréquente encore le lieu pour son port de plaisance. Une splendide demeure de vacances, situé presque au centre
de la ville est à l'abandon. C'est du moins l'impression qu'ont les visiteurs du port de plaisance quand ils
viennent, à partir du 1er juin. En fait, cette dernière est squatté depuis des années par le culte, et ses 
sous sols abritent les reliques du cultes et sert aussi à dissimuler les jeunes femmes et jeunes hommes 
fournies par \gls{ams}

\paragraph{} Mais que fait \gls{osd} des victimes de \gls{osd} ? Ils les livrent simplement aux profonds, il en 
existe une petite communauté, dégénérescente, au large de New York. Cette dernière, fidèle aux règles de la
reproduction de ces créatures, a connu une brutale chute de la natalité à partir de 1925. En effet, les 
profonds avaient atteint leur population "maximum" et, naturellement, comme ils arrivent souvent pour ces 
créatures, la croissance de leur population s'est donc arrêter. En effet, les Profonds étant immortels, se reproduit
sans arrêt auraient des effets plus que néfaste sur leur communauté.

\paragraph{} Malheureuseument pour eux, les assauts du \gls{fbi}, en 1926 et 1927, sur leurs refuges côtiers\footnote{
Voir la nouvelle de HP Lovecraft \emph{Les Ombres de Innsmouth}.} ont largement semé la panique chez les Profonds. Leur
population a mal survécu au stress, et le réflexes naturellement isolationistes de ces créatures, ont détruit des 
communauté entières. Ainsi, 

\subsection{Déroulement de l'affaire}
\subsubsection{Le kidnapping}
\paragraph{} Le 1 mars 1923, à 20h, la nurse du tout jeune file de \gls{clg}, \gls{bgw}, couche le bébé dans son berceau. Vers 
21h30, \gls{chg} entend un bruit qui lui fait penser que quelques choses est tombé dans la cuisine. A 22h, \gls{bgw} réalise que
le bébé n'est plus dans son berceau. Elle se renseigne auprès de Madame Lindbergh, qui vient de sortir de son bain. Réalisant que
le bébé n'est pas non plus avec sa mère, elle se rend auprès de \gls{clg}, qui était dans la bibliothèque juste en dessus de la
nurserie. 
\paragraph{} \gls{clg} se rend à son tour dans la nurserie et constate la disparition de son fils. Alors qu'il fouille la chambre,
il découvre une enveloppe blanche qui a été laissé sur le radiateur, près de la fenêtre. Il s'empare ensuite de son fils et explore 
la maison à la recherche d'intrus. Dans la demi heure qui suivirent, les forces de police locales sont en route pour la demeure,
comme les médias, et l'avocat de \gls{clg}.
\paragraph{} La police fouille les lieux et trouve rapidement une piste boueuse reliant la fenêtre du jeune enfant à un bosquet
où est dissimulée une échelle bricolé mais astucieusement conçu. Si la criminalistique, déjà à l'époque, aurait permis de retirer
beaucoup plus d'éléments à partir de ces quelques traces, le temps pluvieux et la contamination rapide de la scène du crime, n'a 
pas permis de découvrir plus d'éléments.  
\paragraph{} Le \emph{kidnapping}, bien que réalisé un membre de \gls{osd}, n'a fait appel à aucune sorte de magie ou de créatures
du Mythe de Cthulhu. Le cultiste habile a juste pensé à bien masquer ses empreintes dans le sol, une fois l'échelle abandonnée et 
à droguer légèrement le jeune bébé pour s'assurer qu'il ne crie pas pendant sa fuite. Un comparse, aussi membre du culte, l'attendait
non loin dans une voiture.
\subsection{Tout le monde s'emmêle}
\paragraph{} La notoriété de \gls{clg} a effet probablement très désatreux sur l'affaire. Dans la nuit même, \gls{ejh}, réveillé exprès
par un de ses assistants ayant appris la nouvelle, appelle \gls{clg} pour l'informer que le Bureau se tient près à assister l'enquête.
Dans le même temps, \gls{wbs}, le fondateur de la future OSS et de la CIA, arrive et offre aussi ses services. Bientôt, la situation
remonte jusqu'au bureau du président Roosevelt...
\subsection{Gaston Means et la jeune héritière}
 


%http://en.wikipedia.org/wiki/Evalyn\_Walsh\_McLean

%http://en.wikipedia.org/wiki/Gaston\_Means

%http://www.charleslindbergh.com/

% http://news.google.com/newspapers?id=cXAbAAAAIBAJ&sjid=WEsEAAAAIBAJ&dq=gaston%20means&pg=3047%2C5884625
% http://news.google.com/newspapers?id=cXAbAAAAIBAJ\&sjid=WEsEAAAAIBAJ\&dq=gaston\%20means&pg=3047\%2C5884625

% http://en.wikipedia.org/wiki/Saint_Raymond's_Cemetery,_Bronx

% http://en.wikipedia.org/wiki/Virginia_Hall

% http://www.flickr.com/photos/neatnessdotcom/1564083369/sizes/l/

\subsection{timing}

- \gls{ejh} enterre l'affaire dès la découverte du jeune Lindberg le 12 mai:

On May 12, 1932, delivery truck driver William Allen pulled his truck to the side of a road about 4.5 miles (7.2 km) from the Lindbergh home. He went to a grove of trees to relieve himself, and there he discovered the corpse of a toddler
