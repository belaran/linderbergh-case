\documentclass[a4paper,10pt]{article}

\usepackage[utf8x]{inputenc}
\usepackage[french]{babel}
\usepackage{fancyhdr}
\usepackage{graphics}
\usepackage{graphicx}
\usepackage{color}
\usepackage{xcolor}
\usepackage{longtable}
\usepackage{multirow}
\usepackage[pdftex,colorlinks=true,urlcolor=blue]{hyperref}
\usepackage{times}
\usepackage{multicol}
\usepackage{ifthen}
\usepackage{enumitem}
\usepackage[absolute]{textpos}
\usepackage{glossaries} 
%opening
\title{}
\author{}

\newcommand{\glse}[2]
{
  \newglossaryentry{#1}{name={\texttt{#2}},description={TODO\newline Terme présent aux pages : }}
%\include{more/#1}}}
}

\glse{bgw}{Betty Gow}
\glse{ewm}{Evalyn Walsh McLean}
\glse{ejh}{Edgar J.Hoover}
\glse{jafsie}{Jafsie}
\glse{gms}{Gaston Means}
\glse{ams}{Ambrose Morgens}
\glse{clg}{Charles Lingbergh}
\glse{osd}{L'Ordre Esoterique de Dagon de Throg's Neck}
\glse{evk}{Eugene Vander Klei}
\glse{lgt}{Léon G. Turrou}
\glse{john}{John}

\makeglossaries

\begin{document}

\maketitle

\begin{abstract}
Le \emph{Lindbergh Case} est un scénario pour l'Appel de Cthulhu inspirée par l'enquête du même nom 
sur la disparition du fils du célèbre aviateur Charles Lindbergh.


\end{abstract}

\section{Intrigue et contexte}

\subsection{Les rêves eugéniques de Ambrose Morgens}

\paragraph{} Le sorcier \gls{ams} souhaite utiliser la progéniture encore \emph{très fraîche} de 
\gls{clg} pour ses expériences de création de raçe supérieur.

\subsection{L'Ordre Esotérique de Dagon de Throg's Neck}

\paragraph{} Au début des années 1930, \gls{ams} est rentrée en contact avec quelque uns des membres
de l'\gls{osd}, et a commencé avec eux un avantageux commerce. En échange des services de la secte, 
le sorcier leur fournit, sur une base régulière, des \emph{femelles} - comprendre des femmes qu'il a 
kidnappées pour ses propres besoins. Une fois ces dernières rescapés du laboratoire de son comparse, et 
créature, \gls{evk}.

\subsection{Le crime}

\paragraph{} At 8:00 pm on March 1, 1932, the nurse-maid, \gls{bgw}, put 20-month-old Charles Lindbergh, Jr., in his crib. 
She then proceeded to pin the blanket covering him with two large safety pins so as to prevent it from moving while he slept.
At around 9:30 p.m., Col. Lindbergh heard a noise that made him think some slats had fallen off an orange crate in the kitchen. 
At 10:00 p.m., Gow discovered that the baby was missing from his crib. She in turn went to ask Mrs. Lindbergh, who was just coming
 out of the bath, if she had the baby with her. After not finding Charles Lindbergh, Jr., with his mother, 
the nurse-maid then proceeded down stairs to speak with Lindbergh, who was in the library/study just beneath the baby's nursery room 
in the southeast corner of the house. Charles Lindbergh then proceeded up to the nursery to see for himself that his son was not in his crib. 
While surveying the room, he discovered a white envelope had been left on the radiator that formed the window sill. Lindbergh 
proceeded to locate his Springfield rifle and search the rest of the house looking for intruders. Within 30 minutes, the local 
police were on route to the house, as were the media and Lindbergh's attorney. There was a single distinguished footprint and 
indentations discovered a short time later just below the window in the mud due to the rainy and blustery conditions that day 
and into the evening. After the authorities arrived on the scene and began to search the immediate area surrounding the house,s
 a short distance away in a cluster of bushes were found three sections of a smartly designed but rather crude-looking ladder.

http://en.wikipedia.org/wiki/Evalyn\_Walsh\_McLean

http://en.wikipedia.org/wiki/Gaston\_Means

http://www.charleslindbergh.com/

% http://news.google.com/newspapers?id=cXAbAAAAIBAJ&sjid=WEsEAAAAIBAJ&dq=gaston%20means&pg=3047%2C5884625
% http://news.google.com/newspapers?id=cXAbAAAAIBAJ\&sjid=WEsEAAAAIBAJ\&dq=gaston\%20means&pg=3047\%2C5884625

% http://en.wikipedia.org/wiki/Saint_Raymond's_Cemetery,_Bronx

% http://en.wikipedia.org/wiki/Virginia_Hall

% http://www.flickr.com/photos/neatnessdotcom/1564083369/sizes/l/

\section{Scène 1: Arrestation de Gaston Means}

\paragraph{} PJs envoyés arrêtent \gls{gms} suite à la plainte \gls{ewm}, mais ce dernier en profite pour faire le malin 
auprès d'eux, prétendant en savoir bcp sur l'affaire : \emph{Sans moi, vous pouvez déjà oublier tout espoir de retrouver le 
bébé en vie}. Ils l'arrêtent le matin à Washingthon, à son domicile, et se rendent ensuite dans le bureau de \gls{ejh}, 
\emph{le siège du gouvernement}.

\paragraph{} \gls{ejh} sort \gls{lgt} de l'opération car il est connu comme le loup blanc par les différents intervenants et 
qu'il le soupçonne, probablement à raison, de manquer de la subtilité nécessaire pour une telle opération. (PJs rapportent à 
\gls{ejh} à midi qui les informe du rdv secret de ce soir entre \gls{jafsie} et les kidnappeurs.

TODO: décrire le contexte d'un tel rendez vous, la mise en scène de \gls{ejh}.

\section{Scène 2: Mission secret pour le Bureau d'Investigation}

\subsection{Au chat et à la souris, à travers Manhattan}

\paragraph{} Equippé d'une seule voiture de fonction, un banal Fort T, avec dans le coffre un attirail de déguisement, les 
agents arrivent devant la demeure de \gls{jafsie}, après une route épuisante de près de 6h ou 7h\footnote{Test de Conduite pour 
le pilote, sinon -1 point de vie, causé par la fatigue à tous les passages}. Ils arrivent juste à temps, le vieux monsieur, 
clairement nerveux, mais décidé, vérouille sa porte et se prépare à partir.

\subsection{Le Cimetière Saint Raymond}

\paragraph{} Après une délicate filature de \gls{jafsie}, qui a été baladé dans presque tout Manhattan, avant d'être redirigé 
vers le Bronx, les investigateurs arrivent, à quelques centaines de mètres derrière le professeur à la retraite, en vue du 
Cimetière Saint Raymond. Sachant que le premier rendez vous avec les prétendus kidnappeurs a eu lieu dans le cimetière de Woodland,
les agents sentent intuitivement qu'ils parviennent enfin au lieu du rendez vous.

\paragraph{} Pénétrer dans le cimetière à la suite du vieux homme est moins évident que la filature, pourtant déjà peu aisé, qui
les a conduit ici. En effet, la zone est déserte, et \gls{jafsie} a fait déjà un beaucoup infernale, en faisant grincé la porte du 
cimetière. Si quelqu'uns surveille les lieux, suivre, l'air de rien, le vieux professeur ne passera pas inaperçu. 

\paragraph{} Heureusement, même si les investigateurs l'ignorent, la secte n'a rien mis en oeuvre pour surveiller l'arrivée de 
leur intermédiaire. Le vieux professeur ayant respecté les règles du jeux lors du premier rendez vous, ils sont donc désormais un 
peu moins méfiant.

\paragraph{} Quelques soient le stratagème de vos joueurs, leurs personnages devraient arrivées, peu après \gls{jafsie}, au coeur 
du cimetière, et ils ont trouveront refuge derrière une pierre tombale, ironiquement celle d'un Mafiosi mort il y a peu - Joe Cataria, 
pour observer l'échange. Quelques instants après leur arrivée, une église du Bronx sonne au loin minuit - comme quoi ce n'est pas toujours
un cliché, quand émerge de la partie Ouest du cimetière, très boisé, une silouhette. \gls{john}, le contact de \gls{jafsie} vient de 
faire son apparition.

\paragraph{} Tremblant, le vieil homme explique qu'ils n'ont pu réunir que 50 000\$, mais sous interlocuteur ne bronche pas et accepte 
l'argent et lui remet une note lui indiquant que le bébé est sain et sauf à bord d'un bateau, le \emph{Nelly}, à quai dans 
\href{http://en.wikipedia.org/wiki/Martha\%27s\_Vineyard}{Martha's Vineyard}. \gls{jafsie} court prévenir \gls{clg}, qui, lorsqu'ils ne 
trouveront pas le \emph{Nelly}, ira jusqu'à survoler la baie pour tenter de débusquer les kidnappeurs.

\paragraph{} Mais pour les investigateurs, le temps est venu d'honorer la mission secrète que leur a confié \gls{ejh}, suivre \gls{john} et
résoudre l'affaire. Une fois la transaction effectué, \gls{john} regarde le veil homme repartir au pas de course quelques instants, avant 
de lui même se replier vers la même partie boisée du cimetière d'où il a émergé.

\subsection{Retour à la baie de Huson}

\paragraph{} Selon leur discrétion, les agents effectueront une filature, assez complexe, nécessitant quatre jets réussi de Discrétion :
\begin{itemize}
 \item un pour sortir du cimetière, sans se faire repérer par leur cible ;
 \item un autre pour traverser la Hutchison River Parkway - loin d'être une 4 voies d'aujourd'hui à l'époque et peu fréquenté à cette heure,
 toujours sans se faire voir ; 
 \item un troisième jet pour ne pas se faire repérer en traversant les maisons  entre Rohr Pl. et Senger Pl ;
 \item et enfin un ultime jet pour naviguer dans les entrepôt à poissons, dégageant une odeur plus que nauséabonde.
\end{itemize}

\begin{center}
 \includegraphics[scale=0.4]{../handouts/a-la-poursuite-des-profonds-nb.jpg}
\end{center}

\paragraph{} A chaque fois qu'un agent rate son jet de discrétion, \gls{john} peut tenter un test de Vigilance (il a 40\%). Si il réussit, il 
repère les agents et panique : il sort son arme à feu (TODO, 30\%), et effectue un \emph{tir de barrage} pour couvrir sa fuite, maintenant au 
pas de courses, vers le barque qui l'attend près des entrepôt.

\subsection{Abattre John}


\subsection{Perdre sa piste}

\paragraph{} Dans le cimetière, ils verront au moins vers où il aller , donc, reflexes de flics assermentés, ils pourront réveiller et interroger les 
habitants des maisons situés entre Rohr Pl. et Senger Pl. Après quelques maisons, un vieil homme, aussi alcoolic qu'insomniaque, aura vu la
silouhette de la \gls{john} se dirigeant vers les entrepots.

\paragraph{} De là, les agents pour répérer (Trouver Objet Caché) la piste boueuse du suspect et la suivre (Suivre une Piste) jusqu'au bord de 
l'eau.

\subsection{Suivre sa piste}

\paragraph{} Si les agents le suivent sans se faire remarquer, ils le verront monter à bord d'une petit barque, qu'il avait discrètement caché dans les 
arbres. \gls{john} jete le lourd sac de toile noire dans lequel il a placé les 50000\$ remis par\gls{jafsie}, et, le plus silencieusement possible, 
s'engage, à l'aide de rame s'engage sur le \href{http://en.wikipedia.org/wiki/Westchester\_Creek}{Westchester Creek}.

\paragraph{} Si ils suspectent leur présence (plusieurs test de discrétion raté du coté des agents, sans succès de sa part pour les repérer), il 
s'éloignera de la cote, mais se rapprochera de nouveau dès qu'il approchera trop de Clason's Point\footnote{Voir les 
\href{http://www.tentacules.net/index.php?id=1180}{Secrets de New York} pour en savoir plus sur Clason's Point et pourquoi un membre de \gls{osd}
se méfie naturellement de cet endroit}.

\begin{center}
 \includegraphics[scale=0.2]{../handouts/westchester-creek-river.png}
 % westchester-creek-river.png: 1024x683 pixel, 72dpi, 36.12x24.09 cm, bb=0 0 1024 683
\end{center}

\glossarystyle{super}
\printglossaries

\end{document}
